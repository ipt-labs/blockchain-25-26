\section{Розгортання приватної мережі Ethereum на основі WSL}

Потужності мого комп'ютера не є великими (16\,GB RAM, Windows 10, WSL v2.6.3), тому щоб спробувати відтворити блокчейн у себе локально, 
я вирішив для практичної частини взяти мережу \textbf{Ethereum}. Та й загалом причин вибору саме цієї системи декілька:

\begin{itemize}[nosep]
    \item Geth, що працює на базі Linux, повністю сумісний з WSL 2.
    \item Приватна мережа Ethereum вимагає мінімального дискового простору і може працювати з дуже низьким рівнем складності.
    \item Ethereum надає найширший набір функцій для тестування: рахунки, власне майнінг, перекази та розгортання смарт-контрактів.
    \item З 16 ГБ оперативної пам'яті система без проблем підтримує одночасну роботу декількох вузлів Geth.
    \item Багато докладних статей~\cite{pathirana2019, pradeep2020}, тому розібратися саме з Ethereum буде набагато простіше, чим читати документацію інших систем))
\end{itemize}

\subsection{Налаштування середовища і встановлення Geth}

\noindent \textbf{Крок 1:} Оновити (а в кого нема -- встановити) WSL та заінсталити усі необхідні залежності для роботи Geth.

Відкриваємо термінал WSL і запускаємо:

\begin{code}
    # Install Ubuntu distribution for WSL
    wsl --install -d Ubuntu

    # Update package list and upgrade existing packages
    sudo apt update && sudo apt upgrade -y

    # Install required tools
    sudo apt install -y software-properties-common build-essential curl wget
\end{code}

\begin{figure}[H]
    \centering
    \includegraphics[width=0.85\textwidth]{screenshots/screen1.png}
    \caption{Ubuntu version output confirming successful installation}
    \label{fig:screen1}
\end{figure}

\noindent \textbf{Крок 2:} Встановлюємо Geth (Go-Ethereum)

\begin{code}
    # Add Ethereum PPA repository
    sudo add-apt-repository -y ppa:ethereum/ethereum
    sudo apt update

    # Install Geth and related tools
    sudo apt install -y ethereum

    # Verify installation
    geth version
\end{code}

P.S. Можна було і по-старічковому просто скачати бінарник з сайту:) \\ \url{https://geth.ethereum.org/downloads/}

\begin{figure}[H]
    \centering
    \includegraphics[width=0.75\textwidth]{screenshots/screen2.png}
    \caption{\texttt{geth version} output}
    \label{fig:screen2}
\end{figure}

\noindent \textbf{Крок 3:} Створюємо структуру робочої директорії для приватної мережі. Ми будемо запускати два вузли, тому створюємо 
папки для кожного з них, а також папку для логів.

\begin{code}
    # Create project directory
    mkdir -p ~/eth-private/{node1,node2,logs}
    cd ~/eth-private
\end{code}

\subsection{Конфігурація Genesis Block-у}

\noindent \textbf{Крок 4:} Створюємо genesis-файл [як потім вияволося спроба \No 1] для нашої приватної мережі.

\begin{code}
cat > genesis.json << 'EOF'
{
  "config": {
    "chainId": 12345,
    "homesteadBlock": 0,
    "eip150Block": 0,
    "eip155Block": 0,
    "eip158Block": 0,
    "byzantiumBlock": 0,
    "constantinopleBlock": 0,
    "petersburgBlock": 0,
    "istanbulBlock": 0,
    "berlinBlock": 0,
    "londonBlock": 0,
    "terminalTotalDifficulty": 0,
    "terminalTotalDifficultyPassed": true
  },
  "alloc": {},
  "coinbase": "0x0000000000000000000000000000000000000000",
  "difficulty": "0x1",
  "extraData": "0x00",
  "gasLimit": "0x8000000",
  "nonce": "0x0000000000000042",
  "mixhash": "0x0000000000000000000000000000000000000000000000000000000000000000",
  "parentHash": "0x0000000000000000000000000000000000000000000000000000000000000000",
  "timestamp": "0x0"
}
EOF
\end{code}

\begin{figure}[H]
    \centering
    \includegraphics[width=0.75\textwidth]{screenshots/screen3.png}
    \caption{\texttt{genesis.json} \& it's configuration parameters.}
    \label{fig:screen3}
\end{figure}

\noindent \textbf{Крок 5:} Ініціалізуємо блокчейн для ноди 1.

\begin{code}
    geth --datadir ~/eth-private/node1 init ~/eth-private/genesis.json
\end{code}

Все як і очікували: "Successfully wrote genesis state"{}.

\begin{figure}[H]
    \centering
    \includegraphics[width=0.7\textwidth]{screenshots/screen4.png}
    \caption{\texttt{geth init} output confirming successful genesis state creation}
    \label{fig:screen4}
\end{figure}

\subsection{Запуск приватної мережі}

\noindent \textbf{Крок 6:} Запускаємо ноду 1

\begin{code}
    geth --datadir ~/eth-private/node1 \
    --networkid 12345 \
    --http \
    --http.port 8545 \
    --http.api eth,net,web3,miner,admin \
    --nodiscover \
    --port 30303 \
    console 2>> ~/eth-private/logs/node1.log
\end{code}

Щось наплутав з конфігурацією власного Geth ланцюга і Geth повідомляє що "цей ланцюг вже пройшов злиття"{}. Тож 
перестворимо з чистого каталогу без попередньо існуючих даних ланцюжка і у dev mode (мало бути на фото~\ref{fig:screen5}).

\subsection{Спроба 2: Використання dev mode для швидкого розгортання}

Прапор \texttt{-{}-dev} Geth створює self-contained post-Merge development chain з pre-funded account і автоматичним 
блокуванням блоків. Це непоганий спосіб тестування рахунків, транзакцій та смарт-контрактів на сучасних версіях Geth, 
оскільки [як в процесі з'ясувалося] простір імен API \texttt{personal} та команди майнінгу PoW (\texttt{miner.start()}, \texttt{miner.setEtherbase()}) 
були видалені з Geth 1.12+.

\noindent \textbf{Крок 7:} Запускаємо ноду 1 у dev mode і подивимося її дані:

\begin{code}
    geth --datadir ~/eth-private/devnode \
    --dev \
    --http \
    --http.port 8545 \
    --http.api eth,net,web3,admin,debug \
    --http.corsdomain "*" \
    console 2>> ~/eth-private/logs/node1.log
\end{code}

Це відкриє нам консоль Geth JavaScript. Прапорець \texttt{-{}-dev} створює приватний ланцюжок з pre-funded рахунком 
розробника і автоматично видобуває блоки при поданні транзакцій.

\begin{code}
    // List pre-funded account
    eth.accounts

    // Check balance (dev mode pre-funds with max Ether)
    web3.fromWei(eth.getBalance(eth.accounts[0]), "ether")

    // Current block number
    eth.blockNumber
\end{code}

\begin{figure}[H]
    \centering
    \includegraphics[width=0.75\textwidth]{screenshots/screen5.png}
    \caption{Initialization and pre-funded account details}
    \label{fig:screen5}
\end{figure}

\noindent \textbf{Крок 8:} Створюємо другий рахунок для тестування транзакцій.

\begin{code}
    geth --datadir ~/eth-private/devnode account new
    # Enter simple password (e.g., testpass123)
\end{code}

Повернемось до консолі Geth (перший термінал) і переконаймося, що обидва облікові записи відображаються:

\begin{code}
    eth.accounts
\end{code}

\begin{figure}[H]
    \centering
    \includegraphics[width=0.75\textwidth]{screenshots/screen6.png}
    \caption{Account creation output showing the new address and keystore file path}
    \label{fig:screen6}
\end{figure}

\noindent \textbf{Крок 9:} Створимо першу транзакцію.

Переведемо трохи ефіру з "багатого"{} dev рахунку на новостворений рахунок і подивимося, чи блок створюється 
автоматично та подивимося деталі транзакції.

\begin{code}
    // Send 5 Ether from dev account to second account
    eth.sendTransaction({
            from: eth.accounts[0],
            to: eth.accounts[1],
            value: web3.toWei(5, "ether")
        })

    // Dev mode auto-seals a block immediately
    eth.blockNumber

    // Verify recipient balance
    web3.fromWei(eth.getBalance(eth.accounts[1]), "ether")
\end{code}

\begin{figure}[H]
    \centering
    \includegraphics[width=0.75\textwidth]{screenshots/screen7.png}
    \caption{Transaction details and recipient balance}
    \label{fig:screen7}
\end{figure}

\noindent \textbf{Крок 10:} Подивимося деталі блоку та самої транзакції:

\begin{code}
    // Full block information
    eth.getBlock(1)

    // Transaction receipt (gas used, status, etc.)
    eth.getTransactionReceipt("0xTX_HASH_HERE")
\end{code}

Оскільки у нас тестова мережа, то маємо автоматичне підтвердження блоків у момент відправлення транзакції -- у 
реальній мережі нам довелося б чекати, поки валідатор включить її ($\pm$ 12 блоків, або 2-3 хв).

\begin{figure}[H]
    \centering
    \includegraphics[width=0.75\textwidth]{screenshots/screen8.png}
    \caption{Block transaction details and metadata}
    \label{fig:screen8}
\end{figure}

Тут вже можна поглянути на деталі транзакції і самого блоку, куди вона була включена.
\begin{itemize}[nosep]
    \item \texttt{difficulty: 0} -- підтверджує, що це post-Merge PoS chain;
    \item \texttt{gasUsed: 21000} -- це стандартна вартість простого переказу ETH. Кожна транзакція Ethereum має 
        базову вартість 21 000 gas. Взаємодії зі смарт-контрактами коштують дорожче, оскільки EVM повинен виконувати 
        код на додаток до цієї базової вартості.
    \item \texttt{status: "0x1"{}} -- означає, що транзакція була успішною (0x0 означало б невдачу).
    \item \texttt{miner} -- показує на мій dev запис розробника. У PoS це фактично "одержувач комісії"{}, який збирає чайові.
\end{itemize}

\subsection{Спроба 3: Розгортання двох нод у приватній мережі для демонстрації P2P мережі}

Відкриємо Geth на другому терміналі (де ми створювали другий акаунт) і створимо другу ноду для peer з'єднання. Це 
демонструватиме peer-to-peer networking з нашого, описаного в пункті 4, прикладі.

\begin{figure}[H]
    \centering
    \includegraphics[width=0.65\textwidth]{screenshots/screen9.png}
    \caption{Second node initialization with the same genesis file}
    \label{fig:screen9}
\end{figure}

Адреса enode закінчується на :0 замість :30303, а це означає, що Node 1 не шейрить свій порт прослуховування 
правильно, оскільки режим --dev все ж розроблений як одновузлове середовище. І як я одразу про це не прочитав? 
Ну ладно, доведеться закривати ноду 1 і спробувати перестворити, використавши --dev admin API mode для ручного підключення вузлів. 
Порт показує :0 у рядку enode, але вузли все одно можуть підключатися, якщо ми використовуємо шлях IPC (Inter-Process Communication) безпосередньо. Спершу перевіримо фактичний стан прослуховування мережі:

\begin{figure}[H]
    \centering
    \includegraphics[width=0.3\textwidth]{screenshots/screen10.png}
    \caption{Network listening and peer count status}
    \label{fig:screen10}
\end{figure}

net.listening є істинним, але обидва порти мають значення 0. Це підтверджує, що режим --dev приймає ідею мережевого зв'язку, 
але фактично не відкриває жодних портів для з’єднання. Це можна назвати дверима з написом «відкрито», але без ручки, хах. 
Ну що ж, dev не працює, то доведеться робить кастомну приватну мережа з двома відповідними вузлами, щоб продемонструвати P2P-мережі 
та синхронізацію блоків.

\noindent \textbf{Крок 11:} Перестворимо структуру genesis block для приватної мережі і ініціалізуємо обидві ноди:

Перестворимо genesis block, задавши високий поріг \texttt{terminalTotalDifficulty}, який задовольняє вимогу Geth 1.12.x 
після злиття, зберігаючи ланцюг у стані, схожому на стан до злиття. В новіших версіях (після 1.12.x) поле terminalTotalDifficulty 
є невід'ємною складовою генезис-блоку. Розробники повністю відмовилися у підтримці ланцюгів, що працюють на 
PoW і було зробило параметр terminalTotalDifficulty (TTD) критично важливим для ініціалізації або оновлення ланцюгів та 
перейшли на протокол Proof-of-Stake (PoS).

\begin{code}
    cat > ~/eth-private/genesis-pow.json << 'EOF'
    {
    "config": {
    "chainId": 54321,
    "homesteadBlock": 0,
    "eip150Block": 0,
    "eip155Block": 0,
    "eip158Block": 0,
    "byzantiumBlock": 0,
    "constantinopleBlock": 0,
    "petersburgBlock": 0,
    "istanbulBlock": 0,
    "berlinBlock": 0,
    "londonBlock": 0,
    "terminalTotalDifficulty": 999999999999
    },
    "alloc": {},
    "coinbase": "0x0000000000000000000000000000000000000000",
    "difficulty": "0x400",
    "extraData": "0x00",
    "gasLimit": "0x8000000",
    "nonce": "0x0000000000000042",
    "mixhash": "0x00000000000000000000000000000000
    00000000000000000000000000000000",
    "parentHash": "0x0000000000000000000000000000000
    00000000000000000000000000000000000",
    "timestamp": "0x0"
    }
    EOF
\end{code}

Значення 999999999999 \texttt{terminalTotalDifficulty} ланцюг ніколи не досягне шляхом майнінгу, ефективно 
утримуватиме його в режимі до злиття, одночасно задовольняючи обов'язкові вимоги до конфігурації Geth.

\begin{figure}[H]
    \centering
    \includegraphics[width=0.75\textwidth]{screenshots/screen11.png}
    \caption{Contents of updated\texttt{genesis-pow.json}}
    \label{fig:screen11}
\end{figure}

\begin{code}
    geth --datadir ~/eth-private/pow-node1 \
    init ~/eth-private/genesis-pow.json
    geth --datadir ~/eth-private/pow-node2 \
    init ~/eth-private/genesis-pow.json
\end{code}

\begin{figure}[H]
    \centering
    \includegraphics[width=0.75\textwidth]{screenshots/screen12.png}
    \caption{Initialization output from both nodes with matching genesis hashes}
    \label{fig:screen12}
\end{figure}

Обидва вузли ініціалізовані однаковим гешем генезис блоку. Це є надзвичайно важливим, оскільки вузли можуть з'єднуватися 
лише за умови, що вони мають однаковий блок генезису.

\noindent \textbf{Крок 12:} Підключаємо вузли один до одного для формування приватної мережі та перевіряємо зв’язок.

В \textbf{терміналі 1} (консолі Node~1), з'ясовуємо enode URL:

\begin{code}
    admin.nodeInfo.enode
\end{code}

Копіюємо цей рядок enode. У \textbf{терміналі 2} (консолі Node~2) додаємо Node~1 як рівноправного партнера:

\begin{code}
    admin.addPeer("enode://04b6a2...@127.0.0.1:30303")

    // Verify connection
    net.peerCount

    admin.peers
\end{code}

\begin{figure}[H]
    \centering
    \includegraphics[width=0.75\textwidth]{screenshots/screen13.png}
    \caption{Success peer connectionof Node~1 to Node~2's}
    \label{fig:screen13}
\end{figure}

Вихідні дані \texttt{admin.peers} містять важливу інформацію про нашу мережу:
\begin{itemize}[nosep]
    \item \texttt{caps: ["eth/68", "eth/69", "snap/1"]} --- підтримувані версії протоколу.
    \item \texttt{inbound: false} на Node~2 означає, що він ініціював з'єднання; Node~1 показує \texttt{inbound: true} (отримав з'єднання).
    \item Обидва вузли комунікують за допомогою протоколу Ethereum wire (\texttt{eth}) та протоколу синхронізації знімків (\texttt{snap}).
\end{itemize}

\noindent В вересні 2022 року з переходом PoW на PoS процес майнінгу (mining API) ефіру (ETH) було припинено. Тому, на жаль, не 
вийде показати цей процес з подальшою синхронізацією нового блоку на всі ноди навіть у тестовій мережі.

\begin{figure}[H]
    \centering
    \includegraphics[width=0.5\textwidth]{screenshots/screen14.png}
    \caption{Attempting to use deprecated mining commands in Geth console}
    \label{fig:screen14}
\end{figure}

\noindent \textbf{Крок 13:} Проаналізуємо хоч конфігурацію та статус вузлів у мережі.

\begin{code}
    // Full node information
    admin.nodeInfo

    // Protocol configuration
    admin.nodeInfo.protocols
    // Shows chainId, terminalTotalDifficulty, genesis hash

    // Network ID
    net.version

    // Listening status and ports
    net.listening
    admin.nodeInfo.ports
\end{code}

\begin{figure}[H]
    \centering
    \begin{minipage}{0.48\textwidth}
        \centering
        \includegraphics[width=\textwidth]{screenshots/screen15.png}
        \caption{admin.nodeInfo with node \\ configuration and active protocols}
        \label{fig:screen15}
    \end{minipage}
    \hfill
    \begin{minipage}{0.48\textwidth}
        \centering
        \includegraphics[width=\textwidth]{screenshots/screen16.png}
        \caption{admin.nodeInfo.protocols with \\ chain configuration and active EIPs}
        \label{fig:screen16}
    \end{minipage}
\end{figure}

Розділ \texttt{admin.nodeInfo.protocols.eth.config} відображає повну конфігурацію ланцюга, включаючи всі активовані 
EIP. Ці дані підтверджують, що обидва вузли працюють з однаковою конфігурацією приватної мережі та успішно спілкуються 
один з одним.

% ---- PROBLEMS ----

\subsection{Проблеми та їх рішення}
\label{sec:problems}

У процес розгортання блокчейну, я виявив кілька істотних відмінностей між інструкціями знайденими в онлайн-підручниках (більшість з яких написані для Geth~1.9--1.13, 2019-20~р.) та реальністю роботи Geth~1.16.9. 
Проблеми та їхні рішення зібрав у табличку.

\begin{table}[H]
\centering
\begin{tblr}{
        colspec = {X[0.4,c] X[2.5,l] X[2,l] X[2.5,l]},
        hlines, vlines,
        row{1} = {bg=ethpurple!20, font=\bfseries},
        rows = {m},
        cells = {font=\footnotesize},
    }
    \No & Problem                                                                             & Cause                                                                                           & Solution                                                                               \\
    1   & Fatal: \texttt{'terminalTotalDifficulty'} не встановлено в genesis-блоці            & Geth 1.13.x потребує конфігурації після Merge; файли старого зразка без цього поля відхиляються & Додали \texttt{'terminalTotalDifficulty'} до конфігурації genesis                      \\
    2   & \texttt{miner.setEtherbase()} та \texttt{miner.start()} відсутні                    & Geth 1.16.x видалив команди PoW-майнінгу після переходу Ethereum на PoS                         & Використав режим \texttt{--dev} для транзакцій (автоматичне запечатування блоків)      \\
    3   & Dev-режим повідомляє про P2P нульові порти і не дає з'єднатися                      & Режим \texttt{--dev} навмисно вимикає P2P мережу                                                & Використав окрему кастомну мережу для демонстрації P2P                                 \\
    4   & Genesis без \texttt{'terminalTotalDifficulty'} відхиляється навіть для PoW ланцюгів & Geth 1.16.x вимагає конфігурацію post-Merge для всіх ланцюгів                                   & Встанив \texttt{'terminalTotalDifficulty'} на недосяжно високе значення (999999999999) \\
\end{tblr}
\caption{Проблеми, що виникли під час розгортання, та їх вирішення}
\label{tab:problems}
\end{table}

Ці питання ілюструють важливу практичну реальність: перехід Ethereum від Proof-of-Work до Proof-of-Stake (в оновленні «The Merge», вересень 2022 року) 
кардинально змінив робочий процес розгортання. Більшість онлайн-посібників були написані для версій Geth до Merge і зараз частково або повністю застаріли/змінилися 
(конкретні команди, параметри конфігурації ), але основні концепції залишаються актуальними (блоки генезису, виявлення однорангових вузлів, URL-адреси enode, 
консоль JavaScript).

\section{Висновки}

Провівши порівняльний аналіз п'яти основних блокчейн-систем, таких як Ethereum, Bitcoin, Dash, NEO та Litecoin можу підбити 
такі важливі висновки щодо розгортання та взаємозамінності модулів:
\begin{enumerate}
    \item \textbf{Архітектура визначає сумісність.} Системи, що мають спільного предка по кодовій базі (Bitcoin $\to$ Litecoin, 
        Bitcoin $\to$ Dash), демонструють найвищий ступінь структурної схожості та потенціал для повторного використання модулів. 
        Ethereum і NEO, незважаючи на те, що обидві підтримують смарт-контракти, мають фундаментально несумісні архітектури.
    \item \textbf{Консенсус є найменш портативним модулем.} Механізм консенсусу кожної системи тісно пов'язаний з її форматом блоків, 
        логікою переходу між станами та мережевим протоколом. Заміна одного механізму консенсусу на інший вимагає фактично перепроектування 
        всієї системи.
    \item \textbf{Сховище є найбільш портативним модулем.} Використання загальних сховищ ключ-значення (LevelDB, RocksDB) у всіх системах 
        означає, що бекенд сховища можна замінити з мінімальним впливом на поведінку протоколу.
    \item \textbf{Модель стану створює фундаментальний розрив.} Account-based cистеми (Ethereum, NEO) та системи на основі UTXO 
        (Bitcoin, Litecoin, Dash) обробляють транзакції принципово різними способами, що робить сумісність модулів між моделями 
        непрактичним.
    \item \textbf{Приватні мережі Ethereum} є найбільш практичним вибором для освітнього впровадження, пропонуючи багату функціональність 
        (рахунки, транзакції, майнінг(в минулому), смарт-контракти, однорангові мережі) з помірними вимогами до апаратного забезпечення.
\end{enumerate}

\newpage
\selectlanguage{english}
\nocite{*}
\printbibliography
