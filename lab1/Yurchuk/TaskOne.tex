\section{Невеличкий вступ}

Технологія блокчейн кардинально змінила наше уявлення про децентралізовані обчислення, фінансові транзакції та бездовірчі 
системи (trustless systems). З моменту публікації "Bitcoin whitepaper"{}~\cite{nakamoto2008} by Satoshi Nakamoto в 2008 році, 
з'явилося безліч платформ блокчейн, кожна з яких має свої архітектурні рішення, процедури розгортання та операційні характеристики.

У своїй лабораторній, я розглядав і порівнював, здебільшого зазначені в завданнях до лабораторної, особливості розгортання п'яти 
основних систем блокчейну/криптовалюти: \textbf{Ethereum}~\cite{buterin2014, wood2018}, \textbf{Bitcoin}~\cite{nakamoto2008}, 
\textbf{Dash}~\cite{dash2015}, \textbf{NEO}~\cite{neo2018} та \textbf{Litecoin}~\cite{litecoin}. 
Свій аналіз я спрямував більше на практичні аспекти реалізації, розгортання та конфігурації кожної системи, вивченні їх базової 
архітектури, механізмів консенсусу, мережевих рівнів та можливості обміну модулями між ними.

Поки в планах наступне: у розділі 2 описати теоретичні основи кожної системи, далі в розділі 3 доволі детально порівняти їх за різними 
параметрами, розглянути їх взаємозамінність. А далі у розділі 4 планую навести покрокові, як я робитиму, практичне розгортання Ethereum 
(певно найпростіше, що можна взяти) у середовищі на базі WSL за setup-ами від~\cite{pathirana2019, pradeep2020}. І відповідно в кінці підбити підсумки по отриманих результатах.

\section{Огляд систем блокчейн і криптовалют}

\subsection{Bitcoin}

Bitcoin -- це певно перша і найвідоміша криптовалюта, опублікована в 2008 році анонімним девелопером з псевдонімом Сатоші Накамото. 
Вона використовує модель \textbf{UTXO (Unspent Transaction Output)} для відстеження балансів і механізм консенсусу \textbf{Proof-of-Work}, 
заснований на алгоритмі хешування SHA-256.

Ключові особливості Bitcoin включають в себе:
\begin{itemize}[nosep]
    \item \textbf{Жодних смарт-контрактів} (в традиційному сенсі) -- Bitcoin Script навмисно є non-Turing-complete.
    \item \textbf{Block time} -- генерація нового блоку $\approx 10$ хвилин, \textbf{block size} обмежений $\approx 1$--$4$ MB (з SegWit).
    \item \textbf{Загальний обсяг} обмежений 21 мільйоном BTC.
    \item \textbf{Networking} -- використосує особливий мережевий P2P protocol через TCP (port 8333), розповсюдження інформації на основі gossip-based ("пліткування з сусідами").
\end{itemize}

\subsection{Ethereum}

Ethereum -- відкрита розподілена обчислювальна платформа на основі блокчейну, яка підтримує функціональність смарт-контрактів. 
На відміну від Bitcoin, який був розроблений \underline{в першу чергу} як цифрова валюта, Ethereum був задуманий як універсальний 
програмований блокчейн \textcolor{red}{(!)}. Власна криптовалюта блокчейну Ethereum це \textbf{Ether (ETH)}.

\noindent Основні архітектурні особливості Ethereum включають:
\begin{itemize}[nosep]
    \item \textbf{Ethereum Virtual Machine (EVM)} -- віртуальна машина, яка виконує смарт-контракти, з повною функціональністю 
        Тюрінга (це такий а-ля емулятор для виконання програм за скінченний час і пам'ять);
    \item \textbf{State model} -- модель на основі облікових записів діє як банківська книга, відстежуючи баланс і забезпечує 
        можливість укладання складних смарт-контрактів. (в Bitcoin застосовується UTXO model, де рахунки ефективно управляються смарт-контрактами,
        і в якій забезпечується вищий рівень конфіденційності та здійснюється паралельна обробка даних);
    \item \textbf{Protocol of Consensus} -- первісно застосовувався Proof-of-Work (Ethash), але у вересні 22 зробили трансфер на Proof-of-Stake (Casper/Beacon chain);
    \item \textbf{Smart Contracts} -- пишуться мовами Solidity, Vyper та деякими іншими мовами і є EVM-compatible;
    \item \textbf{Gas system} -- обчислювальні витрати вимірюються за допомогою так званих тарифів "gas fees".
\end{itemize}

Ethereum -- це peer-to-peer надбудова над базою інтернет протоколів TCP/IP~\cite{dagan2018}. Кожен вузол запускає копію 
блокчейну та бере участь у валідації та розповсюдженні блоків. Офіційний клієнт \textbf{Geth} (Go-Ethereum) реалізує протокол 
RLC (Remote Procedure Call) Node Discovery Protocol, що базується на DHT (Distributed Hash Table) типу Kademlia для виявлення 
однорангових вузлів (peers)~\cite{dagan2018}.

\begin{figure}[H]
    \centering
    \begin{tikzpicture}[
        node distance=1.2cm and 2cm,
        block/.style={rectangle, draw, fill=ethpurple!15, minimum width=3cm, minimum height=0.8cm, align=center, rounded corners=2pt},
        arrow/.style={-{Stealth[length=3mm]}, thick}
        ]
        \node[block] (app) {DApp / User};
        \node[block, below=of app] (sc) {Smart Contracts (EVM)};
        \node[block, below=of sc] (consensus) {Consensus Layer\\(PoS / formerly PoW)};
        \node[block, below=of consensus] (p2p) {P2P Network\\(RLC / devp2p)};
        \node[block, below=of p2p] (net) {Internet (TCP/UDP)};

        \draw[arrow] (app) -- (sc);
        \draw[arrow] (sc) -- (consensus);
        \draw[arrow] (consensus) -- (p2p);
        \draw[arrow] (p2p) -- (net);
    \end{tikzpicture}
    \caption{Архітектура "шарів"{} Ethereum-а}
    \label{fig:eth-arch}
\end{figure}

\subsection{Dash}

Dash (originally ``Darkcoin'') являє собою fork від кодової бази Bitcoin, але з додатковими функціями конфіденційності 
та управління. В ньому було запроваджено:
\begin{itemize}[nosep]
    \item \textbf{Masternodes} -- мережа другого рівня (second-tier network) з мотивованими (incentivized) вузлами, які забезпечують InstantSend і CoinJoin (PrivateSend).
    \item \textbf{Алгорит гешування X11} -- ланцюжкова послідовність із 11 криптографічних геш-функцій.
    \item \textbf{Система управління фінансами (Treasury system)} -- 10\%  винагороди за блок виділяється на фінансування ідей щодо розвитку (за них голосуються masternodes).
    \item \textbf{Block time} -- генерація нового блоку складає $\approx 2.5$ хвилин.
\end{itemize}

\subsection{NEO}

NEO (колишня назва -- AntShares) китайська блокчейн-платформа, яку частенько називають "китайським ефіром (Ethereum)"{}. 
Вона також підтримує виконання смарт-контрактів, але має значні відмінності по своїй архітектурі:
\begin{itemize}[nosep]
    \item \textbf{Consensus} -- делегована візантійська відмовостійкість (delegated Byzantine Fault Tolerance - dBFT), 
        що забезпечує детермінованість виконання.
    \item \textbf{Dual token model} -- NEO (токен управління, є неподільним) and GAS (токен-утиліта для виконання контрактів).
    \item \textbf{Multi-language support} -- смарт-контракти можуть бути написані на різних мовах програмування, таких як: 
        C\#, Python, Java, Go (з подальшим виконанням у віртуальній машині NeoVM).
    \item \textbf{Інтеграція цифрової ідентичності} та орієнтація на сурове дотримання нормативних вимог.
\end{itemize}

\subsection{Litecoin}

Litecoin був створений Чарлі Лі в 2011 році як "полегшена"{} версія Bitcoin. Це один з найперших форків від Bitcoin, який 
і досі має багато спільного з кодом Bitcoin. Основні особливості:

\begin{itemize}[nosep]
    \item \textbf{Scrypt hashing} -- алгоритм, що вимагає великого обсягу пам'яті (первинно розроблявся для протидії майнінгу ASIC)
    \item \textbf{Block time} -- генерування складає $\approx 2.5$ хвилин (це у 4$\times$ швидше за Bitcoin).
    \item \textbf{Total supply} -- 84 мільйони LTC (у 4$\times$ перевищує обсяг Bitcoin-a).
    \item \textbf{UTXO model} -- така сама структура транзакцій, як у Bitcoin.
    \item \textbf{Test platform} -- Часто стає тестовим майданчиком для функцій Bitcoin-а (наприклад, той же SegWit 
        був активований спочатку на Litecoin).
\end{itemize}

\section{Порівняльний аналіз архітектурних рішень та процедур розгортання}

\subsection{Механізми консенсусу}

Механізм консенсусу є основним модулем, який визначає, як мережа блокчейну досягає згоди щодо стану реєстру. 
Кожен з названих нами п'яти блокчейнів використовують принципово різні підходи:

\begin{table}[H]
    \centering
    \begin{tblr}{
            colspec = {X[1.5,l] X[2,l] X[2,l] X[2,l]},
            hlines, vlines,
            row{1} = {c, bg=ethpurple!20, font=\bfseries},
            rows = {m},
        }
        Блокчейн & Алгоритм гешування & Протокол          & Остаточність                                 \\
        Ethereum & Gasper (PoS)       & Proof-of-Stake    & Probabilistic $\to$ deterministic (2 epochs) \\
        Bitcoin  & SHA-256            & Proof-of-Work     & Probabilistic ($\sim$6 confirmations blocks) \\
        Dash     & X11 + Masternodes  & Hybrid: PoW + PoS & Instant (via ChainLocks)                     \\
        NEO      & dBFT 2.0           & Delegated BFT     & Deterministic (1 block, no forks)            \\
        Litecoin & Scrypt             & Proof-of-Work     & Probabilistic ($\sim$6 confirmations blocks) \\
    \end{tblr}
    \caption{Порівняння протоколів консенсусу}
    \label{tab:consensus}
\end{table}

\subsection{Архітектура мережі та P2P-взаємодія}

Всі п'ять систем працюють як однорангові децентралізовані (peer-to-peer) мережі, побудовані поверх мережі Інтернет. 
Однак їхні протоколи виявлення однорангових вузлів і комунікації значно відрізняються.

\textbf{Ethereum} використовує набір протоколів devp2p, який включає протокол виявлення вузлів RLPx на основі модифікованого 
DHT Kademlia. Кожен вузол ідентифікується за допомогою ідентифікатора \texttt{enode} (хеш SHA3 його відкритого ключа). 
Реалізація Kademlia використовує метрику відстані на основі XOR з 256 сегментами, кожен з яких містить до 16 записів. 
Виявлення однорангових вузлів (peers) використовує чотири типи повідомлень UDP: \texttt{ping}, \texttt{pong}, 
\texttt{findnode} та \texttt{neighbors}. Передача даних відбувається через зашифровані TCP-з'єднання з використанням 
транспортного протоколу RLPx із використанням шифрування ECIES (Elliptic Curve Integrated Encryption Scheme).

\textbf{Bitcoin і Litecoin} використовують простіший P2P-протокол на основі обміну інформацією (gossip-based). Masternodes 
виявляють однорангові вузли за допомогою DNS-посівів і поширення повідомлень \texttt{addr}. Протокол працює через 
TCP (Bitcoin на порту 8333, Litecoin на порту 9333). Структурованого DHT немає; натомість вузли підтримують базу 
даних \texttt{addrman} (address manager) відомих однорангових вузлів.

\textbf{Dash} розширює протокол P2P Bitcoin окремим masternode layer. Masternodes утворюють накладну мережу другого 
рівня з детермінованим порядком, що дозволяє використовувати такі функції, як InstantSend quorums і CoinJoin mixing rounds.

\textbf{NEO} використовує структурований механізм виявлення однорангових вузлів із seed nodes. Консенсус dBFT вимагає 
меншого набору вузлів консенсусу (наразі 7 в основній мережі), які "спілкуються"{} через спеціальний канал консенсусу.

\begin{figure}[H]
    \centering
    \begin{tikzpicture}[
            peer/.style={circle, draw, fill=ethpurple!20, minimum size=0.7cm, inner sep=0pt, font=\tiny},
            master/.style={circle, draw, fill=btcorange!30, minimum size=0.9cm, inner sep=0pt, font=\tiny},
            boot/.style={circle, draw, fill=neogreen!30, minimum size=0.9cm, inner sep=0pt, font=\tiny\bfseries},
            conn/.style={-, thin, gray!60},
            highlight/.style={-, thick, ethpurple}
        ]
        % Ethereum-style network
        \begin{scope}[shift={(-4,0)}]
            \node[boot] (b1) at (0,2) {Bootnode};
            \node[peer] (e1) at (-1.5,0.5) {N1};
            \node[peer] (e2) at (0,0) {N2};
            \node[peer] (e3) at (1.5,0.5) {N3};
            \node[peer] (e4) at (-0.8,-1.2) {N4};
            \node[peer] (e5) at (0.8,-1.2) {N5};

            \draw[conn] (b1) -- (e1);
            \draw[conn] (b1) -- (e2);
            \draw[conn] (b1) -- (e3);
            \draw[conn] (e1) -- (e2);
            \draw[conn] (e2) -- (e3);
            \draw[conn] (e1) -- (e4);
            \draw[conn] (e3) -- (e5);
            \draw[conn] (e4) -- (e5);
            \draw[conn] (e2) -- (e4);
            \draw[conn] (e2) -- (e5);

            \node[below, font=\small\bfseries] at (0,-2) {Ethereum (Kademlia)};
        \end{scope}

        % Bitcoin-style network
        \begin{scope}[shift={(0,0)}]
            \node[peer] (b1) at (0,1.8) {N1};
            \node[peer] (b2) at (-1.3,0.6) {N2};
            \node[peer] (b3) at (1.3,0.6) {N3};
            \node[peer] (b4) at (-0.8,-0.8) {N4};
            \node[peer] (b5) at (0.8,-0.8) {N5};

            \draw[conn] (b1) -- (b2);
            \draw[conn] (b1) -- (b3);
            \draw[conn] (b2) -- (b4);
            \draw[conn] (b3) -- (b5);
            \draw[conn] (b4) -- (b5);
            \draw[conn] (b2) -- (b3);

            \node[below, font=\small\bfseries] at (0,-2) {Bitcoin/LTC (Gossip)};
        \end{scope}

        % Dash-style network (two layers)
        \begin{scope}[shift={(4,0)}]
            \node[master] (m1) at (0,2) {MN};
            \node[master] (m2) at (-1.2,1) {MN};
            \node[master] (m3) at (1.2,1) {MN};
            \node[peer] (d1) at (-1.3,-0.3) {N1};
            \node[peer] (d2) at (0,-0.6) {N2};
            \node[peer] (d3) at (1.3,-0.3) {N3};

            % Masternode layer
            \draw[highlight] (m1) -- (m2);
            \draw[highlight] (m1) -- (m3);
            \draw[highlight] (m2) -- (m3);
            % Regular connections
            \draw[conn] (m2) -- (d1);
            \draw[conn] (m3) -- (d3);
            \draw[conn] (d1) -- (d2);
            \draw[conn] (d2) -- (d3);
            \draw[conn] (m1) -- (d2);

            \node[below, font=\small\bfseries] at (0,-2) {Dash (Two-tier)};
        \end{scope}
    \end{tikzpicture}
    \caption{Топології мереж P2P для різних блокчейнів}
    \label{fig:p2p-networks}
\end{figure}

\subsection{Структури даних та управління станами}

\begin{table}[H]
    \centering
    \begin{tblr}{
            colspec = {X[1.5,l] X[2,l] X[2,l] X[2,l]},
            hlines, vlines,
            row{1} = {bg=ethpurple!20, font=\bfseries},
            rows = {m},
        }
        Блокчейн & Станова модель  & Структура даних               & Cховище           \\
        Ethereum & Account-based   & Modified Merkle-Patricia Trie & LevelDB / Pebble  \\
        Bitcoin  & UTXO            & Merkle Tree                   & LevelDB           \\
        Dash     & UTXO (extended) & Merkle Tree + MN lists        & LevelDB / RocksDB \\
        NEO      & Account-based   & Merkle Tree + state root      & LevelDB           \\
        Litecoin & UTXO            & Merkle Tree                   & LevelDB           \\
    \end{tblr}
    \caption{Станові моделі і структури даних в блокчейнах}
    \label{tab:datastructures}
\end{table}

Фундаментальна різниця між account-based системами (Ethereum, NEO) та UTXO (Bitcoin, Dash, Litecoin) проявляється в 
в процесі розгортання та взаємозамінності модулів. У системах UTXO кожна транзакція споживає попередні виходи та створює 
нові. У account-based системах глобальний стан явно відстежує залишки на рахунках та cховище контрактів.

\subsection{Можливості смарт-контрактів}

\begin{table}[H]
    \centering
    \begin{tblr}{
            colspec = {X[1.5,l] X[1.5,l] X[2,l] X[2,l]},
            hlines, vlines,
            row{1} = {bg=ethpurple!20, font=\bfseries},
            rows = {m},
        }
        Блокчейн & Віртувальне середовище & Мови програмування    & Здібності блокчейну       \\
        Ethereum & EVM                    & Solidity, Vyper       & Turing-complete           \\
        Bitcoin  & Bitcoin Script         & Script opcodes        & Non-Turing-complete       \\
        Dash     & ---                    & ---                   & No native smart contracts \\
        NEO      & NeoVM                  & C\#, Python, Java, Go & Turing-complete           \\
        Litecoin & Bitcoin Script         & Script opcodes        & Non-Turing-complete       \\
    \end{tblr}
    \caption{Порівняння можливостей блокчейнів щодо смарт-контрактів}
    \label{tab:smartcontracts}
\end{table}


\subsection{Deployment Parameters and Requirements}

В наступній таблиці хочу навести основні параметри розгортання для запуску повного вузла блокчейну:

\begin{table}[H]
    \centering
    \begin{tblr}{
            colspec = {X[1.5,l] X[1.2,c] X[1.2,c] X[1.2,c] X[1.8,c]},
            hlines, vlines,
            row{1} = {bg=ethpurple!20, font=\bfseries},
            rows = {m},
        }
        Блокчейн & RAM (min) & ROM (min)    & Default Port & Primary Client    \\
        Ethereum & 8 GB      & $\sim$1 TB+  & 30303        & Geth / Nethermind \\
        Bitcoin  & 2 GB      & $\sim$600 GB & 8333         & Bitcoin Core      \\
        Dash     & 4 GB      & $\sim$40 GB  & 9999         & Dash Core         \\
        NEO      & 4 GB      & $\sim$20 GB  & 10333        & neo-cli (C\#)     \\
        Litecoin & 2 GB      & $\sim$120 GB & 9333         & Litecoin Core     \\
    \end{tblr}
    \caption{Параметри розгортання mainnet у 2024--2025 роках}
    \label{tab:deployment}
\end{table}

\subsection{Конфігурація генезис-блоку}

Будь який блокчейн починається з так званого \textbf{генезис-блоку (genesis block)}. У приватних/тестових розгортаннях його конфігурація визначає початкові параметри мережі. Нижче наведено приклад для Ethereum (\texttt{genesis.json}):
\begin{code}
    {
    "config": {
    "chainId": 12345,
    "homesteadBlock": 0,
    "eip150Block": 0,
    "eip155Block": 0,
    "eip158Block": 0,
    "byzantiumBlock": 0,
    "constantinopleBlock": 0,
    "petersburgBlock": 0,
    "istanbulBlock": 0,
    "berlinBlock": 0,
    "londonBlock": 0
    },
    "alloc": {},
    "coinbase": "0x0000000000000000000000000000000000000000",
    "difficulty": "0x400",
    "extraData": "0x00",
    "gasLimit": "0x8000000",
    "nonce": "0x0000000000000042",
    "mixhash": "0x0000000000000000000000000000000000000000000000000000000000000000",
    "parentHash": "0x0000000000000000000000000000000000000000000000000000000000000000",
    "timestamp": "0x0"
    }
\end{code}

\noindent Основними важливими полями є:
\begin{itemize}[nosep]
    \item \texttt{chainId} -- унікальний ідентифікатор, що відрізняє створювану приватну мережу від основної мережі (chainId=1).
    \item \texttt{difficulty} -- контролює складність майнінгової головоломки (mining puzzle complexity); нижчі значення дозволяють швидший майнінг у тестових середовищах.
    \item \texttt{gasLimit} -- максимальний дозволений обсяг гасу на блок, що встановлює верхню межу для виконання контракту.
    \item \texttt{alloc} -- дозволяє ініціалізувати (робити pre-funding) рахунків валютою Ether при їх створенні.
    \item Поля \texttt{*Block} вказують, під яким номером блоку активуються різні EIP (пропозиції щодо вдосконалення Ethereum) та хард-форки. Встановлення всіх значень на 0 дозволяє активувати всі функції з самого початку.
\end{itemize}

\paragraph{} Для порівняння, генезисний блок Bitcoin hard-coded в клієнтському програмному забезпеченні і не може бути налаштований 
таким же чином. Тому при розгортанні тестової мережі Bitcoin або мережі regtest параметри модифікуються під час компіляції 
за допомогою констант або прапорців конфігурації (\texttt{-regtest}, \texttt{-testnet}).

\subsection{Створення блоків і час їх генерації}

\begin{figure}[H]
    \centering
    \begin{tikzpicture}
        \begin{axis}[
                ybar,
                bar width=0.8cm,
                ylabel={Block Time (seconds)},
                symbolic x coords={Ethereum, Bitcoin, Dash, NEO, Litecoin},
                xtick=data,
                ymin=0,
                ymax=700,
                nodes near coords,
                nodes near coords align={vertical},
                every node near coord/.append style={font=\small},
                width=0.85\textwidth,
                height=7cm,
                grid=major,
                grid style={dashed, gray!30},
                xticklabel style={rotate=0, anchor=north},
            ]
            \addplot[fill=ethpurple!60] coordinates {
                    (Ethereum, 12)
                    (Bitcoin, 600)
                    (Dash, 150)
                    (NEO, 15)
                    (Litecoin, 150)
                };
        \end{axis}
    \end{tikzpicture}
    \caption{Середній час генерації блоків (в секундах)}
    \label{fig:blocktimes}
\end{figure}

\subsection{Пропускна здатність}

\begin{figure}[H]
    \centering
    \begin{tikzpicture}
        \begin{axis}[
                ybar,
                bar width=0.8cm,
                ylabel={Transactions per Second (TPS)},
                symbolic x coords={Ethereum, Bitcoin, Dash, NEO, Litecoin},
                xtick=data,
                ymin=0,
                ymax=1200,
                nodes near coords,
                nodes near coords align={vertical},
                every node near coord/.append style={font=\small},
                width=0.85\textwidth,
                height=7cm,
                grid=major,
                grid style={dashed, gray!30},
            ]
            \addplot[fill=btcorange!60] coordinates {
                    (Ethereum, 30)
                    (Bitcoin, 7)
                    (Dash, 56)
                    (NEO, 1000)
                    (Litecoin, 56)
                };
        \end{axis}
    \end{tikzpicture}
    \caption{Приблизна пропускна здатність транзакцій базового рівня (TPS) для різних блокчейнів}
    \label{fig:tps}
\end{figure}
