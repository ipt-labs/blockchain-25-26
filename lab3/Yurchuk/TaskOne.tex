\section{Вступ}
\label{sec:intro}


Швидке поширення децентралізованих фінансів (DeFi) запровадило принципово нову парадигму в галузі безпеки програмного 
забезпечення. На відміну від традиційних веб-додатків, де централізований сервер обробляє бізнес-логіку за брандмауером, 
децентралізовані додатки (dApps) виконують критично важливі фінансові операції на блокчейні. Кожна функція смарт-контракту 
є видимою для всіх учасників, кожна транзакція транслюється перед підтвердженням, а розгорнутий код не може бути виправлений 
без складних стратегій міграції~\cite{atzei2017survey}.

Станом на 2024 рік протоколи DeFi сумарно управляють сотнями мільярдів доларів у Total Value Locked (TVL), проте інциденти 
безпеки залишаються тривожними, а злами -- частими. Згідно з щорічним звітом Immunefi, збитки, пов'язані з криптовалютою, 
від хакерських атак та зловживань перевищили 1,8 мільярда доларів лише у 2024 році~\cite{immunefi_report}. Ці втрати є 
наслідком поєднання традиційних веб-вразливостей у інтерфейсах фронтенду з новими векторами атак, характерних для блокчейну, 
таких як reentrancy, flash loan manipulation, and Maximal Extractable Value (MEV)~\cite{daian2020flash}.

Проект Open Web Application Security Project (OWASP) є галузевим стандартом безпеки для багатьох веб-додатків з 
2001 року~\cite{owasp_about}. Його список вимог "Top 10"{} містить узгоджену класифікацію найкритичніших ризиків безпеки, 
що впливають на веб-додатки~\cite{owasp_top10_2021}. Однак, традиційна структура OWASP була розроблена з урахуванням 
централізованих архітектур клієнт-сервер, не враховує унікальні, і тим паче нові, загрози для систем на основі блокчейну.

Протягом лабораторної, я хотів би розглянути відмінності між встановленими "золотими"{} стандартами веб-безпеки в далекому 2001 та 
новими вимогами до безпеки dApp. Я обирав Uniswap для аналізу децентралізованого протоколу, оскільки він охоплює всі 
рівні сучасного стеку dApp: незмінні смарт-контракти в ланцюжку, алгоритми маршрутизації поза ланцюжком, веб-інтерфейс, 
що обслуговується через традиційну інфраструктуру DNS, механізми управління на базі токена UNI та функціональність 
оракула, який встановлює ціни. Багато з цих складових використовується сотнями інших протоколів~\cite{uniswap_v3_whitepaper}. 
Ця багатошарова архітектура робить його ідеальним кандидатом для спостереження як традиційних, так і специфічних для блокчейну 
вимог безпеки.

\newpage % Forced
\section{OWASP Top 10:2021 --- десять найкритичніших місць безпеки у веб-додатках}
\label{sec:owasp}

\begin{figure}[H]
    \centering
    \begin{tikzpicture}[
            scale=0.7,
            transform shape,
            node distance=0.35cm,
            catbox/.style={
                    draw, rounded corners=4pt, minimum width=13.5cm,
                    minimum height=0.85cm, font=\sffamily\small,
                    text=white, align=left, text width=12.8cm,
                    inner xsep=6pt
                }
        ]
        \node[catbox, fill=owaspRed!95] (a01)
        {\textbf{A01} \enspace Broken Access Control \hfill CWE-200, CWE-284, CWE-285};
        \node[catbox, fill=owaspRed!85, below=of a01] (a02)
        {\textbf{A02} \enspace Cryptographic Failures \hfill CWE-259, CWE-327, CWE-331};
        \node[catbox, fill=owaspRed!75, below=of a02] (a03)
        {\textbf{A03} \enspace Injection \hfill CWE-79, CWE-89, CWE-078};
        \node[catbox, fill=owaspOrange!90, below=of a03] (a04)
        {\textbf{A04} \enspace Insecure Design \hfill CWE-209, CWE-256, CWE-501};
        \node[catbox, fill=owaspOrange!75, below=of a04] (a05)
        {\textbf{A05} \enspace Security Misconfiguration \hfill CWE-16, CWE-611};
        \node[catbox, fill=owaspYellow!85, text=black, below=of a05] (a06)
        {\textbf{A06} \enspace Vulnerable \& Outdated Components \hfill CWE-1104};
        \node[catbox, fill=owaspYellow!70, text=black, below=of a06] (a07)
        {\textbf{A07} \enspace Identification \& Authentication Failures \hfill CWE-287, CWE-384};
        \node[catbox, fill=dappBlue!65, below=of a07] (a08)
        {\textbf{A08} \enspace Software \& Data Integrity Failures \hfill CWE-502, CWE-829};
        \node[catbox, fill=dappBlue!50, below=of a08] (a09)
        {\textbf{A09} \enspace Security Logging \& Monitoring Failures \hfill CWE-778};
        \node[catbox, fill=dappBlue!40, text=black, below=of a09] (a10)
        {\textbf{A10} \enspace Server-Side Request Forgery (SSRF) \hfill CWE-918};
    \end{tikzpicture}
    \caption{OWASP Top 10:2021 категорії, відсортовані за ступенем небезпеки з відповідними ідентифікаторами CWE (ступенем ризику)}
    \label{fig:owasp_top10_overview}
\end{figure}

OWASP Top~10 це інформаційний документ, який відображає загальний консенсус щодо найкритичніших ризиків безпеки 
веб-додатків, його періодично оновлювлюють. У випуску 2021 року (останнє велике оновлення), було внесено значні зміни порівняно 
з попереднім (2017 рік), зокрема додано три повністю нові категорії та методологію на основі даних, що базується на аналізі 
понад 500{,}000 реальних додатків~\cite{owasp_top10_2021}. Кожна категорія коротко характеризується переліком типових 
вразливостей (CWE)~\cite{cwe_mitre}.

\subsection{A01:2021 --- Broken Access Control}
\label{sec:a01}

Порушення контролю доступу піднялося з п'ятої позиції в 2017 році на перше місце в 2021 році. Це відображає поширеність 
поширеність проблеми у сучасних додатках. Ця категорія охоплює будь-які порушення в застосуванні політик, що обмежують 
дії користувачів до дозволених. Типові прояви включають: порушення принципу мінімальних привілеїв, обхід перевірок 
контролю доступу шляхом модифікації URL-адрес, запитів API або внутрішнього стану додатка, а також незахищені прямі 
посилання на об'єкти [insecure direct object references] (IDOR), коли введені користувачем дані безпосередньо 
посилаються на внутрішній ресурс без перевірки авторизації.

У контексті веб-додатків контроль доступу зазвичай забезпечується серверним проміжним програмним забезпеченням, моделями 
контролю доступу на основі ролей [role-based access control] (RBAC) та механізмами управління сеансами. Стандарт перевірки 
безпеки додатків OWASP [application security verification standard] (ASVS) надає детальні вимоги до контролю доступу на 
трьох рівнях забезпечення~\cite{owasp_asvs}. Наприклад, ASVS~V4 вимагає, щоб усі рішення щодо контролю доступу приймалися 
на надійному сервері і не могли бути обійдені маніпуляціями на стороні клієнта.

Критично важливим для безпеки dApp є те, що традиційні моделі контролю доступу передбачають наявність надійного сервера. 
У смарт-контрактах "сервером"{} виступає сам блокчейн, а контроль доступу здійснюється за допомогою модифікаторів Solidity, 
шаблонів власності та відображень ролей, що зберігаються в змінних стану контракту~\cite{openzeppelin_contracts}.

\subsection{A02:2021 --- Cryptographic Failures}
\label{sec:a02}

У минулому ця категорія називалася "Розкриття конфіденційних даних"{} (Sensitive Data Exposure), але тепер вона більше 
переорієнтована на помилки, пов'язані з реалізацією криптографічних механізмів. Вона охоплює: використання слабких 
або застарілих криптографічних алгоритмів, неналежне управління ключами, передачу даних у вигляді відкритого тексту, 
невиконання вимог TLS та використання слабких генераторів псевдовипадкових чисел.

OWASP наголошує, що конфіденційні дані обов'язково мають класифікуватися відповідно до нормативних та бізнес-вимог, 
а також що відповідний криптографічний захист повинен застосовуватися як під час зберігання, так і під час їх передачі. 
Стандарт рекомендує використовувати AES-256 для симетричного шифрування, RSA-2048+ або Curve25519 для асиметричних операцій 
та SHA-256 (або SHA-512) для гешування, одночасно відмовляючись від використання MD5, SHA-1 та DES~\cite{owasp_top10_2021}.

Для систем блокчейну криптографічні примітиви є фундаментальними, а не допоміжними. Ethereum використовує еліптичну криву 
\texttt{secp256k1} для всіх підписів транзакцій і \texttt{keccak256} для гешування. Криптографічна помилка на цьому рівні 
була б катастрофічною, хоча такі помилки частіше трапляються в програмному забезпеченні криптогаманця або на рівні управління 
ключами, ніж у самому протоколі.

\subsection{A03:2021 --- Injection}
\label{sec:a03}

Вклинювальні атаки (Injection attacks) відбуваються, коли шкідливі дані надсилаються інтерпретатору як частина команди або 
запиту. Класичними прикладами є SQL injection, NoSQL injection, OScommand injection та міжсайтовий скриптинг (XSS). 
Основні рекомендації OWASP включають: використання параметризованих запитів і заздалегіть підготовлених висловлювань, застосування 
перевірки вхідних даних за допомогою списків дозволених, контекстне екранування вихідних даних та використання заголовків Content 
Security Policy (CSP) для зменшення ризику XSS. Сучасні веб-фреймворки значною мірою автоматизували багато з цих засобів захисту, 
але неправильні конфігурації та власні шляхи коду залишаються вразливими.

У контексті dApp традиційне ін'єктування є актуальним на рівні фронтенду, тоді як на рівні смарт-контрактів існує 
унікальна форма ін'єкцій: ретельно розроблені вхідні дані транзакцій, які використовують недоліки коду, зокрема 
через зловмисні функції зворотного виклику під час міжконтрактних викликів.

\subsection{A04:2021 --- Insecure Design}
\label{sec:a04}

Нова категорія з 2021 року це небезпечний дизайн. Віе відображає фундаментальну зміну філософії OWASP: від реактивного 
виявлення вразливостей до проактивної безпечної архітектури. Ця категорія стосується недоліків у дизайні та архітектурі 
додатка, які неможливо виправити на рівні реалізації.

Основні практики, рекомендовані OWASP, включають: моделювання загроз під час проектування (за допомогою STRIDE, PASTA 
або подібних фреймворків), встановлення безпечних шаблонів проектування та еталонних архітектур, написання випадків 
зловживання поряд з випадками використання та включення вимог безпеки з найраніших етапів проектування.

Для децентралізованих систем незахищений дизайн, мабуть, є найкритичнішою категорією, оскільки смарт-контракти є 
незмінними після розгортання. Дефект дизайну, виявлений після розгортання, не може бути виправлений; він вимагає 
цілої міграції на новий контракт, що передбачає складні управлінські рішення та потенційну втрату стану протоколу.

\subsection{A05:2021 --- Security Misconfiguration}
\label{sec:a05}

Ця категорія охоплює будь-які неправильно налаштовані засоби контролю безпеки, включаючи: стандартні облікові дані, 
увімкнені непотрібні функції, надмірно відкриті хмарні сховища, відсутні заголовки безпеки, докладні повідомлення 
про помилки, що розкривають внутрішні деталі, та застаріле програмне забезпечення з відомими вразливостями на 
рівні конфігурації.

OWASP рекомендує процес зміцнення, що включає в себе видалення невикористовуваних функцій, регулярний перегляд конфігурацій, 
впровадження інфраструктури як коду для забезпечення відтворюваних безпечних конфігурацій та надсилання директив безпеки, 
таких як заголовки \texttt{X-Content-Type-Options}, \texttt{Strict-Transport-Security} та \texttt{X-Frame-Options}.

Для інтерфейсів децентралізованих додатків (dApp) неправильна конфігурація є особливо небезпечною, оскільки інтерфейс 
є основним засобом, за допомогою якого користувачі підписують і передають транзакції. Неправильно налаштована політика 
безпеки вмісту може дозволити введення скрипту, який непомітно замінює параметри транзакції до того, як користувач 
верифікує її зі свого гаманця.

\subsection{A06:2021 --- Vulnerable and Outdated Components}
\label{sec:a06}

Безпека програмного забезпечення стала одним з головних питань. Ця категорія стосується ризику використання застарілих 
бібліотек, фреймворків або інших програмних компонентів із відомими вразливостями. Типова веб-програма покладається 
на сотні транзитивних залежностей, кожна з яких може містити вразливості, які можна використати.

OWASP з цього приводу рекомендує вести перелік компонентів програмного забезпечення (SBOM), постійно перевіряти оновлення 
компонентів на наявність відомих вразливостей у базах даних \\ (CVE/NVD), використовувати інструменти сканування залежностей 
(OWASP Dependency-Check, Snyk, npm audit) та встановити політику управління виправленнями.

У сфері смарт-контрактів це означає використання перевірених бібліотечних контрактів (таких як перевірені на 
практиці реалізації OpenZeppelin) замість власного або форкованого коду, а також забезпечення актуальності версій 
компіляторів та відсутності відомих багів~\cite{openzeppelin_contracts, solidity_docs}.

\subsection{A07:2021 --- Identification and Authentication Failures}
\label{sec:a07}

Ця категорія раніше так само мала іншу назву -- "Порушення автентифікації"{} і охоплювала слабкі місця у перевірці 
особистості користувачів, управлінні сеансами та обробці облікових даних. До типових проблем належать: слабка політика 
щодо паролів, вразливість до наповнення облікових даних, фіксація сеансів, неналежне закриття, в т.ч. неактивних, 
сеансів та відсутність багатофакторної автентифікації.

OWASP рекомендує впроваджувати багатофакторну автентифікацію, застосовувати політику сильних паролів, обмежувати 
швидкість автентифікаційних кінцевих точок та дотримуватися практик безпечного управління сеансами, таких як повторне 
генерування ідентифікаторів сеансів після входу в систему~\cite{owasp_asvs}.

В екосистемі блокчейнів аутентифікація принципово відрізняється від звичного нам розуміння. Користувачі аутентифікуються 
за допомогою власних криптографічних підписів зі своїх гаманців, а не за допомогою комбінації імені користувача та 
пароля. "Сесій"{}, у традиційному, розумінні не існує. Однак сам процес підключення гаманця створює унікальні 
проблеми з аутентифікацією, зокрема щодо абстракції облікового запису EIP-4337 та інтеграції апаратного гаманця~\cite{eip4337}.

\subsection{A08:2021 --- Software and Data Integrity Failures}
\label{sec:a08}

Також ще одна нова категорія була додана у 2021 році. Вона охоплює збої, пов'язані з кодом та інфраструктурою, що не захищають 
від порушень цілісності. Включає: незахищені процеси з CI/CD, механізми автоматичного оновлення без перевірки підпису, 
десеріалізацію ненадійних даних та використання залежностей з ненадійних джерел.

OWASP особливо підкреслює ризик компрометації pipelines build (як приклад -- атаки SolarWinds) і рекомендує: перевіряти 
цифрові підписи на \underline{всіх} компонентах програмного забезпечення, використовувати Subresource Integrity (SRI) 
для сторонніх скриптів та впроваджувати процеси перегляду змін конфігурації CI/CD.

Для dApps ця категорія теж має критичні наслідки. Фронтенд зазвичай завантажує пакети JavaScript з CDN або шлюзу IPFS, 
і цілісність цих пакетів безпосередньо визначає, чи взаємодіють користувачі з справжнім протоколом або з його зловмисним 
клоном. Атака на DNS Uniswap у 2022 році продемонструвала цей ризик на практиці~\cite{uniswap_dns_attack}.

\subsection{A09:2021 --- Security Logging and Monitoring Failures}
\label{sec:a09}

Недостатнє протоколювання (logging) та моніторинг були підвищені з нижчої позиції, задля відображенння їх важливості 
у виявленні інцидентів та реагуванні на них. OWASP рекомендує вести журнали всіх подій автентифікації, збоїв контролю 
доступу, збоїв перевірки вхідних даних на стороні сервера та аномалій бізнес-логіки; забезпечувати, щоб журнали містили 
достатній обсяг контекст без запису конфіденційних даних; встановлювати порогові сповіщення; та проводити регулярний 
перегляд журналів.

Системи блокчейну мають тут унікальну перевагу: сам блокчейн є незмінним, публічно доступним для перевірки журналом усіх операцій, 
що змінюють його стан. Однак компоненти поза ланцюгом (фронтенд-сервери, шлюзи API, служби маршрутизації) все ще потребують традиційної 
інфраструктури реєстрації, а події в ланцюзі повинні активно індексуватися та моніторитися.

\subsection{A10:2021 --- Server-Side Request Forgery (SSRF)}
\label{sec:a10}

SSRF виникає, коли веб-додаток звертається до віддаленого ресурсу без перевірки URL-адреси, наданої користувачем. 
Зловмисники можуть зловживати цим для доступу до внутрішніх служб, сканування внутрішніх мереж або викрадення даних 
за допомогою переприв'язки DNS.

Рекомедації для захисту є наступними: очищати та перевіряти всі введені клієнтом URL-адреси, застосовувати списки 
дозволених доменів і протоколів (whitelists), вимикати перенаправлення HTTP та сегментувати доступ до мережі, 
щоб мінімізувати радіус ураження від використання SSRF~\cite{owasp_top10_2021, nist_sp800_53}.

В архітектурі dApp ризики SSRF зосереджені в сервісах бекенду, які взаємодіють з вузлами RPC блокчейну, 
серверами метаданих токенів та шлюзами IPFS. Вразливість SSRF в API маршрутизації може дозволити зловмиснику 
перенаправити виклики RPC на зловмисний вузол, який повертає сфальсифікований стан блокчейну.

\section{Архітектура протоколу Uniswap}
\label{sec:uniswap_arch}

Uniswap це найбільша децентралізована біржа (DEX) на Ethereum за обсягом торгів. Вона дозволяє здійснювати обмін 
токенами без посередників за допомогою моделі Automated Market Maker (AMM), де ліквідність забезпечується 
користувачами в пулах на ланцюжку, а не через традиційну книгу замовлень. Протокол пройшов чотири основні 
версії: v1 (2018), v2 (2020), v3 (2021) і v4 (2023), кожна з яких впроваджувала все більш досконалі механізми~\cite{uniswap_v2_whitepaper, uniswap_v3_whitepaper}.

\subsection{Рівні архітектури Uniswap}

Архітектура Uniswap охоплює кілька рівнів, кожен з яких має свої особливості безпеки та поверхні атаки. На 
діаграмі~\ref{fig:uniswap_layers} можна спостерігати цю багаторівневу структуру:

\begin{figure}[H]
    \centering
    \begin{tikzpicture}[
        layer/.style={
                draw, rounded corners=6pt, minimum width=13cm,
                minimum height=1.6cm, font=\sffamily, align=center,
                text width=12cm, inner sep=8pt
            },
        arrow/.style={-{Stealth[length=6pt]}, thick, gray!70},
        node distance=0.5cm
        ]
        % Layers from top to bottom
        \node[layer, fill=uniPink!12, draw=uniPink!60] (fe)
        {\textbf{Layer 5: User Interface (Front-End)}\\
            \small React SPA served via Cloudflare $\cdot$ Wallet Connect SDK $\cdot$ app.uniswap.org};

        \node[layer, fill=dappBlue!10, draw=dappBlue!50, below=of fe] (api)
        {\textbf{Layer 4: Off-Chain Routing \& API}\\
            \small Uniswap Auto Router $\cdot$ Subgraph Indexer (The Graph) $\cdot$ Token Lists};

        \node[layer, fill=secGreen!10, draw=secGreen!60, below=of api] (router)
        {\textbf{Layer 3: On-Chain Router Contracts}\\
            \small SwapRouter $\cdot$ UniversalRouter $\cdot$ Permit2};

        \node[layer, fill=owaspOrange!12, draw=owaspOrange!60, below=of router] (core)
        {\textbf{Layer 2: Core Protocol Contracts}\\
            \small UniswapV3Factory $\cdot$ UniswapV3Pool $\cdot$ NonfungiblePositionManager};

        \node[layer, fill=gray!10, draw=gray!50, below=of core] (chain)
        {\textbf{Layer 1: Ethereum Blockchain}\\
            \small EVM execution $\cdot$ Consensus $\cdot$ Mempool $\cdot$ Block production};

        % Arrows
        \draw[arrow] (fe) -- (api);
        \draw[arrow] (api) -- (router);
        \draw[arrow] (router) -- (core);
        \draw[arrow] (core) -- (chain);

        % Side labels
        \node[font=\sffamily\scriptsize, text=gray, rotate=90, anchor=south] at ([xshift=-7.2cm]$(fe)!0.5!(api)$) {HTTPS/WSS};
        \node[font=\sffamily\scriptsize, text=gray, rotate=90, anchor=south] at ([xshift=-7.2cm]$(api)!0.5!(router)$) {JSON-RPC};
        \node[font=\sffamily\scriptsize, text=gray, rotate=90, anchor=south] at ([xshift=-7.2cm]$(router)!0.5!(core)$) {EVM calls};
        \node[font=\sffamily\scriptsize, text=gray, rotate=90, anchor=south] at ([xshift=-7.2cm]$(core)!0.5!(chain)$) {State transitions};

    \end{tikzpicture}
    \caption{Архітектура протоколу Uniswap.}
    \label{fig:uniswap_layers}
\end{figure}


\subsection{Core Smart Contracts (Layer 2)}

За рівень 1, як працює блокчейн з протоколами консенсусу, ми говорили в попердніх лабораторних, тому одразу перейдемо 
і зосередимося на рівні 2. "Серце"{} Uniswap v3 складається з двох основних контрактів~\cite{uniswap_v3_whitepaper}:

\textbf{UniswapV3Factory} -- це одноразовий(?) [singleton] контракт, який розгортає нові екземпляри пулу ліквідності. 
Кожен пул параметризується парою токенів і рівнем комісії (0,01\%, 0,05\%, 0,3\%, або 1\%). Фабрика використовує 
\texttt{CREATE2} для детермінованих адрес пулів, що дозволяє будь-якому учаснику передбачити адресу пулу до його розгортання.

\textbf{UniswapV3Pool} implements the concentrated liquidity AMM. Unlike v2 (where liquidity is distributed uniformly across the entire price range $[0, \infty)$), v3 allows liquidity providers (LPs) to allocate capital within specific price ranges called ``ticks.'' This design dramatically improves capital efficiency but introduces complexity in position management, fee accounting, and oracle calculations.

\textbf{UniswapV3Pool} реалізує AMM (Automated Market Maker) з сконцентрованою ліквідністю. На відміну від v2 
(де ліквідність розподілялася рівномірно по всьому діапазону цін $[0, \infty)$), v3 дозволяє постачальникам 
ліквідності (liquidity providers, LP) розподіляти капітал у межах певних діапазонів цін, які називаються "ticks". 
Така конструкція значно покращує ефективність використання капіталу, але ускладнює управління позиціями, облік 
комісій та обчислення оракулів.

Формула добутку, що лежить в основі AMM Uniswap, виражається дуже простою рівністю:

\begin{equation*}
    x \cdot y = k
\end{equation*}

\noindent де $x$ та $y$ представляють резерви двох токенів in pool, а $k$ -- інваріант, який повинен підтримуватися 
(або збільшуватися) після кожного обміну. У v3 ця формула діє в межах кожного активного діапазону тіків із використанням 
віртуальних резервів, при цьому реальна ліквідність концентрується відповідно до позицій LP.

По наступному шматочку кода можна бачити критичну сигнатуру функції \texttt{swap} з UniswapV3Pool:

\begin{minted}{solidity}
function swap(
    address recipient,
    bool zeroForOne,
    int256 amountSpecified,
    uint160 sqrtPriceLimitX96,
    bytes calldata data
) external override noDelegateCall returns (
    int256 amount0,
    int256 amount1
);
\end{minted}

Модифікатор \texttt{noDelegateCall} є механізмом безпеки, що гарантує неможливість виконання pool's logic за 
допомогою \texttt{DELEGATECALL} із зовнішнього контракту, який міг би маніпулювати контекстом зберігання.

\subsection{Router and Periphery Contracts (Layer 3)}

Користувачі не взаємодіють безпосередньо з контрактами. Натомість вони направляють транзакції через периферійні контракти:

\textbf{SwapRouter02} обробляє багатоступеневі swaps (наприклад, ETH $\rightarrow$ USDC $\rightarrow$ DAI) і забезпечує 
захист від прослизання (slippage) за допомогою перевірки мінімальної суми виходу. 

\textbf{UniversalRouter} -- це нещодавнє доповнення, яке об'єднує обміни ERC-20, покупки NFT та затвердження Permit2 в 
одну транзакцію.

\textbf{Permit2} -- це менеджер затвердження токенів, який замінює застарілу модель ERC-20 \texttt{approve} на дозволи 
на основі підпису, з обмеженим терміном дії та обмеженою сумою~\cite{uniswap_docs}.

\subsection{Off-Chain Components (Layers 4--5)}

Auto Router -- це позаланцюговий алгоритм, який обчислює оптимальний маршрут обміну між усіма пулами Uniswap і рівнями 
комісій, а також між ліквідністю v2 і v3. Він використовує індексатор підграфіків (The Graph) для запиту поточного 
стану пулів. Фронтенд -- це односторінковий додаток React (SPA), розміщений на традиційній веб-інфраструктурі 
(Cloudflare CDN) з резервними розгортаннями на IPFS.

Важливо (!) -- фронтенд обробляє підключення гаманця, побудову параметрів транзакції та потік підписання користувача. 
Будь-яке порушення цього рівня дозволяє зловмиснику змінювати параметри транзакції до підписання користувачем, ефективно 
викрадаючи кошти, незважаючи на безпеку самих смарт-контрактів.
